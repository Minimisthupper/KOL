\documentclass[12pt]{article} 
\usepackage[utf8]{inputenc}
\usepackage[
backend=biber,
style=alphabetic,
sorting=ynt
]{biblatex}
% add bibliography-file
\addbibresource{bibliography.bib}
% used to fix some 'overfull hbox in bibliography'
\emergencystretch=1em

\usepackage{times}
\usepackage[breaklinks]{hyperref}
\usepackage[onehalfspacing]{setspace}
\pagenumbering{arabic}
\usepackage{geometry}
 \geometry{
 a4paper,
 left=25mm,
 right=30mm,
 top=25mm,
 bottom=20mm,
 }
%German-specific commands
%--------------------------------------
\usepackage[ngerman]{babel}
\usepackage{csquotes}
%--------------------------------------
%Hyphenation rules
%--------------------------------------
\usepackage{hyphenat}
\hyphenation{Mathe-matik wieder-gewinnen}
%--------------------------------------
%Inhaltverzeichnis Benennung
%--------------------------------------
\renewcommand{\contentsname}{Inhaltsverzeichnis}
%--------------------------------------

\title{KOL Floyd Erdmann}
\author{Floyd Erdmann}
\date{October 2021}

\begin{document}
\setlength{\baselineskip}{5mm}
\maketitle
\onehalfspacing

\newpage

\tableofcontents

\newpage
\section{Einleitung}


\section{Krankheitserreger}
\subsection{Überblick der wichtigsten Krankheitserreger beim Menschen}
Der Überbegriff „Krankheitserreger“, in der Medizin auch „Keime“ oder „Infektionserreger“, steht für subzelluläre Erreger oder Mikroorganismen, welche in anderen Organismen gesundheitsschädigende Abläufe verursachen. Beim Menschen wird man grundsätzlich zwischen den vier am meisten verbreiteten Erregerarten unterschieden:
\begin{itemize}
    \item \subsubsection{Bakterien}
    Bakterien bilden die einfachste Lebensform auf dem Planeten Erde. Sie sind einzellige Kleinstlebewesen und werden den Prokaryoten, den Lebewesen ohne Zellkern zugeordnet.\footnote{Vgl. \cite{Nagel2021}} Mit einem Durchmesser von 0,1 bis 700 Mikrometer\footnote{Vgl. \cite{BZgA2021}} sind sie um ein Vielfaches größer als Viren. Infektionskrankheiten, welche durch Bakterien ausgelöst werden können, sind beispielsweise Salmonellose, Tuberkulose, Keuchhusten oder Blutvergiftung.
    \item \subsubsection{Viren}
    Da Viren im Gegensatz zu Bakterien weder aus einer eigenen Zelle bestehen noch einen eigenen Stoffwechsel betreiben, werden sie von Virologen nicht zu den Lebewesen zugeordnet, sondern als organische Struktur bezeichnet. Aufgrund ihrer Fähigkeit zur Evolution werden sie jedoch trotzdem als „dem Leben nahestehend“ betrachtet. Viren bestehen nur aus ihrem Erbgut, der DNA oder RNA, sowie aus Proteinen und brauchen daher eine Wirtszelle um sich zu vermehren.
    \item \subsubsection{Parasiten}
    Als Parasiten bezeichnet man Lebewesen, welche auf oder in einem Organismus einer anderen Art, dem Wirt, leben und/oder Nahrung von diesem beziehen. Der Wirt bleibt dabei in der Regel am Leben, wird jedoch in seiner Gesundheit oder seinem Wohlbefinden geschädigt. Krankheiten, welche durch Parasiten am Wirt hervorgerufen werden, bezeichnet man als Parasitosen. 
    \item \subsubsection{Pilze}
    Pilze sind eigenständige Lebewesen mit einem sehr vielseitigen Erscheinungsbild. Sie können sowohl in der Umwelt als auch in einem Wirt existieren und leben auf der menschlichen Haut, indem sie sich von abgestorbenen Zellen ernähren. Krankheiten, welche unter bestimmten Bedingungen wie beispielsweise Schwächung des Imunsystems durch Pilze hervorgerufen werden, bezeichnet man medizinisch als Mykosen.
    
\end{itemize}
Eine Ansteckung mit einem Krankheitserreger nennt man Infektion. Hierbei bilden die Parasiten eine Ausnahme, da der Befall eines Organismus mit einem Parasiten als Infestation bezeichnet wird.

\subsection{Übertragungswege der Krankheitserreger}
Die Art und Weise und Wahrscheinlichkeit einer Übertragung von Krankheitserregern ist durch die Diversität der Eigenschaften dieser stark beeinflusst. Außerdem führt nicht jede Übertragung eines Keims auch zu einem symptomatischen Krankheitsverlauf, da das Imunsystem Erreger teils ohne auftretende Nebenwirkungen abtöten kann. Beim Menschen wird grundsätzlich in folgende Übertragungswege unterschieden:
\begin{itemize}
    \item \subsubsection{Tröpfcheninfektion}
    Bei der Infektion durch Sekret-Tröpfchen wird Atemwegssekret durch Niesen, Husten oder Sprechen in die Umgebung freigesetzt. Diese Tröpfchen können Krankheitserreger enthalten und eine gesunde Person anstecken, falls diese durch die Schleimhäute von Mund, Nase oder Augen in den Körper aufgenommen werden. Durch diesen Übertragungsweg werden vorallem Atemwegserkrankungen wie Keuchhusten weitergegeben. Auch Infektionen der Grippe und des Covid-19 Virus erfolgen meist über eine Tröpfcheninfektion. Maßnahmen zum Vorbeugen einer solchen Infektion sind das Minimieren von Kontaktpersonen, gute Hygiene und gegebenenfalls das Tragen einer Schutzmaske.
    Fälschlicherweise wird die „Übertragung durch die Luft“ häufig als eigenständiger Übertragungsweg erwähnt, wobei es sich um eine Tröpfcheninfektion durch sehr kleine, teilweise stundenlang in der Luft schwebende Sekret-Tropfen handelt. Wie schnell Tröpfchen und Aerosole in der Luft schweben oder absinken ist dabei stark von der Luftbewegung, der Luftfeuchtigkeit und der Belüftung abhängig.\footnote{Vgl. \cite{Rki21}}
    \item \subsubsection{Schmierinfektion}
    Eine andere Weitergabemöglichkeit von Krankheitserregern ist die Kontaktübertragung mittels Berührung. Dies kann sowohl von Mensch zu Mensch als auch über den Kontakt von verunreinigten Gegenständen erfolgen. Der Aufbau des Erregers und die Beschaffenheit der Oberfläche spielen hierbei eine große Rolle, da sie entscheidend für die Lebensdauer von Viren und Bakterien auf Objekten sind.
    \item \subsubsection{Übertragung durch Lebewesen}
    Infektionskrankheiten wie Borreliose können von einem infizierten Tier, welches als Träger des Krankheitserregers dient, auf einen Menschen übertragen werden. Das Tier muss bei diesem Übertragungsweg nicht zwangsweise erkrankt sein. Häufig auftretende Beispiele sind Borreliose durch Zeckenstiche, Tollwut und Ebola.
    \item \subsubsection{Übertragung durch Lebensmittel}
    Wenn Lebensmittel, welche von Krankheitserregern befallen sind, von Menschen verzehrt werden, kann dies eine Erkrankung auslösen. Häufig handelt es sich dabei um eine Diarrhoe. In Deutschland sind vorallem Bakterien wie Salmonellen und Listerien und Viren wie Noroviren und Rotaviren für eine Infektion durch Lebensmittel verantwortlich.
    \item \subsubsection{Übertragung durch Wasser}
    Krankheitserreger können über das Wasser durch verschiede Wege in den menschlichen Körper gelangen. Eine Infektion über das Trinkwassser kann auch der Übertragung durch Lebensmittel zugeordnet werden, allerdings ist eine Übertragung von Krankheiten wie Campylobacter und EHEC auch durch das Baden in verunreinigten Gewässern möglich und stellt somit eine eigene Übertragungsvariante dar.
\end{itemize}
Eine Übertragung von Krankheitserregern erfolgt also durch körperliche Nähe und den daher erfolgenden Austausch von Erregern über Körperflüssigkeiten, über verunreinigte Gegenstände, Lebensmittel und Wasser und über andere Lebewesen.
\subsection{Vermehrung von Krankheitserregern}
Im Kontext dieser Komplexarbeit befasst sich der Autor ausschließlich mit der Verbreitung von Viren, da das Hauptaugenmerk dieser Ausarbeitung auf dem Zusammenhang von Viren mit der Globalisierung gelegt wird.
\subsubsection{Vermehrung von Viren}
Um sich zu vermehren, benötigen Viren eine Wirtszelle eines lebenden Organismus, da der Virus selbst zu keinem Stoffwechselvorgang fähig ist. Die Vermehrung kann in zwei Arten unterschieden werden. Beim lytischen Zyklus, welcher stets den Tod der befallenen Zelle als Folge hat, schleust das Virus seine Nukleinsäure in die Wirtszelle ein, während beim lysogenischen Zyklus die Replikation des Virus meist ohne den Zelltod stattfindet. Bei dieser Art der Vermehrung wird die genetische Information des Virus in das Wirtsgenom integriert, was die Entstehung eines Provirus zur Folge hat. Um den Vermehrungsprozess zu beginnen, heften sich Viren bei beiden Übertragungsarten an die Zellmembran um in der anschließenden Phase der Penetration das Viruserbgut in die Zelle freizusetzen. Dies ist über die Fusionierung, dem Verschmelzen der Virushülle mit der Zellmembran der Wirtszelle und anschließenden Abgabe des Erbgutes ins Zellinnere, oder den der Endozytose, dem Absenken des Virus in die Zellmembran der Wirtszelle durch welches ein Vesikel entsteht, das den Virus ins Zellinnere transportiert, möglich. Zellmaschinerien, welche normalerweise das Erbgut der Wirtszelle ablesen, haben nun auch Zugriff auf das Viruserbgut, womit eine wichtige Vorrraussetzung für die Virusvermehrung gegeben ist. In der darauffolgenden Phase produziert die Wirtszelle anhand des Viruserbguts neue Virusproteine und Viruserbgut und stellt somit alle einzelnen Bestandteile eines Virus in großer Zahl her. Eine Polio-infizierte Zelle stellt beispielsweise circa 1000 neue Viren pro Zelle her. Die Einzelteile des neu produzierten Virus sind in der Lage sich von selbst zu einem kompletten Virus zusammenzulagern. Den Abschluss einer erfolgreichen Virusvermehrung stellt die Freizetzung dieser dar. Durch das Schädigen der Zelle mit dessen Tod als Folge und dadurch efolgende Freisetzung des Virus, der Knospung, bei welcher der Virus mit Abschnitten der Zellmembran abgeschnürt wird, oder der Sekretion, dem Abschnüren in das Innere der Organellen und anschließende Freisetzung des Virus durch Vesikel, wird die Vermehrung des Virus vollendet.
\section{Globalisierung der letzten 100 Jahre}
\subsection{Zur Globalisierung}
Globalisierung bezeichnet einen Prozess, in dem weltweite Verflechtungen in vielen Bereichen, wie Wirtschaft, Politik, Kultur, Umwelt und Kommunikation, zwischen Individuen, Gesellschaften, Institutionen und Staaten zunehmen. Als wesentliche Gründe gelten der technische Vortschritt mit wichtigen Innovationen in Produkt- und Prozessbereichen, sowie in Kommunikations- und Transporttechnologien, die ordnungspolitische Grundorientierung mit Entscheidungen und Maßnahmen zur Liberalisierung des Welthandels und das Bevölkerungswachstum in 229 von 235 Staaten seit 1950. Kennzeichnend für Globalisierung ist die Zunahme an Verbindungen zwischen Gesellschaften und Problembereichen. Dies geschieht unter numerischer Zunahme, qualitativer Intensivierung und einer räumlichen Ausdehnung. Es kann in folgende Arten der Globalisierung unterschieden werden:
\begin{itemize}
    \item \subsubsection{Ökonomische Globalisierung}
    Mit einer Steigerung des weltweit statistisch erfassten Warenexportes um mehr als das 19fache im Zeitraum von 1960 bis 2017 ist die Globalisierung in starkem Maße ökonomischer Natur. Trotz enormem Zuwachs des weltumspannenden Handels und Expansion der Transnationalen Unternehmen durch große Kapitalströme, ermöglicht durch Abbau von Regulierungen im Wirtschafts- und Finanzbereich, kann man die Wirtschaft noch nicht als vollständig global bezeichnen, da lediglich 20 Prozent der Güter und Dienstleistungen international gehandelt werden. Zusätzlich sind derzeit nur circa 30 Prozent der Weltbevölkerung in die Weltwirtschaft integriert. Der Begriff „Welthandel“ ist auch deshalb etwas irreführend, da der Einfluss ganzer Kontinente, beispielsweise Afrika und Südamerika, sich auf einen einstelligen Prozentsatz beläuft. Entwicklungsländer in diesen Kontinenten sind aufgrund von politischer Instabilität, mangelhafter Rechtssicherheit und unzureichender Infrastruktur meist vom Globalisierungsprozess ausgeschlossen. Weitherhin ist die Globalisierung anhand der verstärkten Abkopplung der Finanzmärkte von der realwirtschaftlichen Entwicklung zu erkennen, wodurch es zu gehäuften kurzfristigen Kapitalanlagen der spekulativen Art kommt und Finanzmärkte zum „Handlungsort der neuen Gestaltung der Welt werden“ \footnote{Vgl. \cite{Renz2001}}. Zur Globalisierung der Wirtschaft zählt außerdem die Veränderung in Transport und Personenverkehr. So stieg trotz steigender Anzahl der Containerschiffe im Weltseehandel auch die Kapazität derer in den letzten 40 Jahren um 2500 Prozent\footnote{Vgl. \cite{Keller2021}}. Diese Entwicklung des Transportwesens wirkt sich nicht nur positiv, beispielsweise eine Zunahme von Arbeitsplätzen, aus sondern bringt auch schwerwiegende Probleme, zum Beispiel ökologischer Natur, mit sich. Mit der Ausweitung der Zug-, Automobil- und Luftverkehrsnetze weiten sich Personenverkehr und Tourismus grenzüberschreitend aus. Ein letzter wichtiger Aspekt der ökonomischen Globalisierung, welcher in den letzten Jahrzehnten enormes Ausmaß angenommen hat, ist die Kommunikation und das Internet. Mit einer seit 1960 Verzehnfachung der Telefonanschlüsse im weltweiten Telefonnetz und einem Anstieg der Internetnutzer um fast 50 Prozent allein in den letzten 15 Jahren\footnote{Vgl. \cite{Rabe2021}} ist es einfacher global zu kommunizieren als je zuvor.
    \item \subsubsection{Politische Globalisierung}
    % overfull hbox, Lösung siehe Antwort 2: https://tex.stackexchange.com/questions/111948/what-is-a-overfull-hbox-9-89561pt-too-wide
    \begin{sloppypar}
    Die Globalisierung der Politik resultiert aus den Folgen der wirtschaftlichen und kulturellen Gloablisierung. Durch Hürden in den Problemfeldern Wirtschaft, Natur und Sicherheitspolitik ist eine globale Kooperation in vielen Fällen unumgänglich, da die begrenzten nationalen Möglichkeiten zur Problembewältigung nicht ausreichen. Zwei mögliche Lösungswege werden diskutiert um diesen Schwierigkeiten entgegenzuwirken. Der Versuch Globalisierung zu verhindern und zu einem gewissen Grad zurückzudrehen um durch Regionalismus eine Gegenmacht zu bilden und den an den Markt verlorenen Einfluss zurückzugewinnen, stellt den ersten Lösungsweg dar. Ein anderer Ansatz ist der Versuch, globalpolitische Strukturen und Regelwerke zu kreieren um künftige Herausforderungen zu bewältigen. Drei Ebenen auf welche sich die politische Globalisierung bezieht, sind die Zunahme von internationalen Verträgen, Vereinbarungen und ein Anstieg der internationalen Öffentlichkeit mit einer auf globale Ereignisse ausgerichtete Berichterstattung. Die Globalisierung der Politik sorgt einerseits für internationale Zusammenarbeiten und Organisationen wie UNO, WELTBANK und IWF kreiert aber andererseits einen steigenden Konkurrenzkampf zwischen den einzelnen Nationalstaaten.
    \end{sloppypar}
    \item \subsubsection{Gesellschaftliche und Kulturelle Globalisierung}
    \begin{sloppypar}
    Auch die gesellschaftliche und kulturelle Globalisierung hängt stark von der wirtschaftlichen Globalisierung ab. Die Diffusion unterschiedlicher Kulturen, welche während der Zeit des Kolonialismus durch Inbesitznahme fremder Territorien und Unterwerfung der ansässigen Bevölkerung stattgefunden hat und heutzutage vor allem durch Tourismus und moderne Massenkommunikationsmittel fortgeführt wird, steht dabei im Vordergrund. Um trotz wirtschaftlicher Globalisierung größtmöglichen Profit zu erzielen, versuchen transnationale Unternehmen Teil der jeweilig vorherrschenden Kultur zu werden. Die Auswirkungen der Globalisierung der Kultur äußern sich beispielsweise durch eine stetig steigende Anzahl an bikulturellen Partnerschaften und nach UN-Schätzungen circa 281 Millionen Migranten weltweit.\footnote{Vgl. \cite{WMR2020}} Eine weitere Folge ist in der Globalisierung der Sprache zu erkennen. So breitet sich der Gebrauch der englischen Sprache stetig aus\footnote{Vgl.\cite{Sprache2016}} und Exonyme verschwinden zunehmend aus Nationalsprachen.
    \end{sloppypar}
    \item \subsubsection{Ökologische Globalisierung}
    Einhergehend mit positiven Auswirkungen der Globalisierung, wie einem weltweit steigenden Pro-Kopf Einkommen\footnote{Vgl. \cite{Lammar2013}}, stetig steigenden Lebenserwartungen\footnote{Vgl. \cite{Radtke2021}} und besser ausgebauter Infrastruktur, verschärfen sich jedoch auch die ökologischen Probleme. Zwar steigt aufgrund der Globalisierung auch die Wahrnehmung für global auftretende Schäden und Industrieländer schaffen Umweltstandards und strenge Umweltgesetze\footnote{Vgl. \cite{WikipGlobWirtsch}}, diese können allerdings von Schwellen- und Entwicklungsländern nicht eingehalten werden. Aufgrund des Wettbewerbdrucks der wirtschaftlichen Globalisierung erlangen vordergründig Staaten Vorteile, welche Umweltstandards und Auflagen niedrig ansetzen oder sogar bestehende Regulierungen aufheben. Der Welthandel und der damit verbundene Transportaufwand und die Warenherstellung hat außerdem zu einem enormen Energieverbrauch geführt. Laut einer Studie von „Our World in Data“ belief sich der Anteil von Energiegewinnung an der Gesamtheit der globalen Treibhausgasemissionen auf über 70 Prozent\footnote{Vgl. \cite{Ritchie2021}}. Da sich Nationalstaaten bei Zusammentreffen wie den Klimagipfeln sehr schwer tun notwendige Schritte einzuleiten und Abkommen zu vereinbaren um zukünftige Umweltkatastrophen abzuwenden, bildeten sich in den letzten Jahrzehnten diverse Nichtregierungsorganisationen wie Greenpeace und Amnestie International, welche mithilfe ihres Einflusses Druck auf Regierungen ausüben um verschiedene Ziele zu forcieren\footnote{Vgl. \cite{Ehrlich1966} }.
\end{itemize}
\subsection{Chancen und Risiken der Globalisierung}
\subsubsection{Chancen}
Eine der wichtigsten Chancen der Globalisierung ist die Durchsetzung der Menschenrechte in einem internationalen Rahmen. Zwar gibt es bis heute Staaten, in welchen es immer noch zu schwerwiegenden und teils systematischen Menschenrechtsverstößen kommt, jedoch hat sich der globale Zustand dank Einfluss der Industrieländer auf Entwicklungsländer und modernen Kommunikationsmitteln deutlich verbessert.\footnote{Vgl. \cite{Lohrmann}}\footnote{Vgl. \cite{WikipMenschRechte}} 
Eine weitere Gelegenheit für Fortschritt durch Globalisierung lässt sich im potenziellem Weltfrieden erkennen. Auf Basis der Staatsform Demokratie in vielen entwickelten Ländern und internationalen Bündnissen und Verträgen entsteht ein friedliches Verhältnis zwischen Nationen. Um dies zu erreichen muss die Außenpolitik demokratisch gesinnter Staaten jedoch mithelfen, Demokratie in anderen Ländern zu erreichen. Das dieses Ziel noch lange nicht erreicht ist, kann man an politischen Spannungen wie beispielsweise zwischen China und Taiwan erkennen.
Aus ökonomischer Sicht stellt die Globalisierung eine große Chance dar, da durch einen Anstieg im Handel und verstärkte Arbeitsteilung Armut durch Arbeitsplätze bekämpft werden kann. Dies lässt sich durch den Anstieg der weltweiten Exporte im Warenhandel seit 1950\footnote{Vgl. \cite{Urmersbach2021}} und steigendem Wohlstand mit höheren Lebenserwartungen. Diese Entwicklung beinhaltet auch den Rückgang der Anzahl von in absoluter Armut lebender Menschen\footnote{Vgl. \cite{Roser2016}}.
Eine positive Entwicklung stellt die vermehrte Toleranz und der Austausch unterschiedlicher Kulturen dar. Trotz der erwähnten Diffusion, bleiben mit den jeweiligen Kulturen einhergehende Praktiken und Formen des Ausdrucks erhalten und werden sogar in andere Kulturen übernommen. So müssen Filmproduktionsgesellschaften beispielsweise drauf achten, Filme für ein heterogenes Publikum zu produzieren und Diskriminierung in keiner Weise zu unterstützen um größtmögliches Profitpotential zu erreichen. Somit führt Globalisierung indirekt zu einer zunehmend toleranteren Welt. Eine weitere Form des Umdenkens lässt sich im Zusammenhang mit der ökologischen Globalisierung erkennen, da durch moderne Kommunikationsmittel ein „planetares Bewusstsein“\footnote{Vgl. \cite{WikipGlobWirtsch}} begünstigt wird.
\subsubsection{Risiken}
Globalisierungsskeptiker kritisieren die globale Integration heftig und sehen eine große Bedrohung in der vom Markt beherrschten Welt.\footnote{Vgl. \cite{Renz2001}} Ein schwerwiegender Kritikpunkt ist dabei, dass die Globalisierung für eine Vernachlässigung von Menschen- und Arbeitnehmerrechten, ökologischen Standards und Demokratie verantwortlich sei und nur Märkte und Geschäftsbeziehungen fördere.\footnote{Vgl. \cite{IntGB2010}} Das Individuum habe auf die Entwicklung aufgrund steigendem Einfluss des Marktes wenig bis keinen Einfluss. Dies sorgt vor allem bei ökologische denkenden Aktivisten für großen Unmut.
Ein großes Risiko für die bisherige Grundidee des Wohlfahrstaates ist der ansteigende Modernisierungsdruck. Der Steuersenkungswettlauf um in der modernen Wirtschaft mithalten zu können sorgt für minimale Staatseinnahmen, wodurch wohlfahrtstaatliche Leistungen unbezahlbar werden. Die stetige Zunahme der Lebenserwartung und damit einhergehend längere Rentenzahlungen sowie Behandlungs- und Medikamentkosten benötigten eine Entwicklung in die entgegengesetzte Richtung.

Die zurzeit wohl populärste negative Folge und eins der größten Risiken für die Zukunft der Menscheit sind die Umweltschäden infolge Globalisierung. Sorgeerregernd ist dabei das Tempo der Verhandlungsfortschritte bei internationalen Klimakonferenzen. So gab es bereits 1979 eine erste Weltklimakonferenz\footnote{Vgl. \cite{Weltklimakonferenzen}}, doch erst im Jahr 2015 bei der in Paris stattgefundenen Klimakonferenz kam es zum Vertragsabschluss. Die Klimaerwärmung als Folge der Globalisierung spielt in dieser Arbeit eine besondere Rolle, da vor allem die Ausbreitung von vektorgebundenen Krankheiten wie Malaria und Leishmaniasis direkt und indirekt vom Klima beeinflusst ist.\footnote{Vgl. \cite{Ebert2005}}

\section{Spanische Grippe und Coronavirus}
Vor 104 Jahren verbreitete sich eine Influenza-Pandemie, welche mit 50 Millionen Todesfällen zur verheerendste Pandemie der Geschichte wurde. %Inwiefern sich Verlauf und die Verbreitung des Subtypes A/H1N1, der "spanischen Grippe", dem aktuellen Covid-19-Virus ähnelt wird der Autor in diesem Vergleich offenlegen.
Trotz einer Weltbevölkerung von nur 1,8 Milliarden Menschen und einer nicht annähernd so fortgeschrittenen Globalisierung wie 2021 forderte die Spanische Grippe somit deutlich mehr Todesopfer als das derzeit kursierende Coronavirus.
\subsection{Zur Spanischen Grippe}
Als virueller Abkömmling des Influenzevirus, Subtyp A/H1N1, infizierte das Virus zwischen 1918 und 1920 in drei Wellen etwa 500 Millionen Menschen weltweit.\footnote{Vgl. \cite{Span}} Mit einer Letalität, der Tödlichkeit einer Erkrankung, von circa 5 bis 10 Prozent, war die Sterblichkeit infizierter Menschen deutlich höher als bei Erkrankungen durch andere Influenza-Erreger. Zusätzlich zur hoher Sterblichkeitsrate wies der Erreger andere Besonderheiten wie ein großes Ansteckungsrisiko und, im Gegenstz vorherigen Influenzaviren durch welche vorallem Kleinkinder und ältere Menschen gefährdet sind, eine erhöhte Sterblichkeitsrate bei 20- bis 40-jährigen Menschen. 

\subsubsection{Krankheitsverlauf und Symptome}
Trotz von anderen Influenzaviren abweichenden Merkmalen ähnelten die spezifischen Symptome der Spanische Grippe sehr jener anderen Influenza-Erkrankungen. So begann der Krankheitsverlauf sehr plötzlich durch ein ausgeprägtes Krankheitsgefühl im ganzen Körper mit Kopfschmerzen, Glieder-, Kreuz- und Rückenschmerzen, aber auch Müdigkeit, Abgeschlagenheit und Antriebsschwäche.\footnote{Vgl. \cite{Span}} Während es bei Infektionen der ersten Welle vorallem zu milden Krankheitsverläufen mit genannten Symptomen kam, wirkte sich die zweite Welle häufig deutlich schneller und gravierender aus. Infizierte starben nur wenige Stunden nach Erstsymptomen\footnote{Vgl. \cite{Hanan2021}}, aber auch Überlebende hatten oft noch Wochen Probleme mit starker Müdigkeit und chronischer Erschöpfung. Auch die Anzahl der Infizierten mit Lungenentzündungen, hämorrhagischem Fieber und Zyanose, bläulich-schwarze Verfärbung der Haut durch Mangel an Sauerstoff, stieg mit der zweiten Welle des Virus stark. 
\subsubsection{Ausbreitung und Verlauf}
Die Spanische Grippe verbreitete sich, wie auch Influenza im Allgemeinen, durch die Weitergabe per Tröpfcheninfektion und mit einer Verringerung der Infektiösität ab 22 Grad Celcius entspricht auch die Umweltstabilität des Virus dem der Allgemeinen Influenzaerkrankung. Die erste, im Frühjahr 1918 in Nordamerika beginnenden, Ausbreitungswelle wies mit einer Letalität von 5 Toden pro 1000 Menschen\footnote{Vgl. \cite{Jeff06}} keine außergewöhnlich hohe Tödlichkeit auf, verbreitete sich jedoch ungewöhnlich schnell. Diese große Ausbreitungsgeschwindigkeit der ersten Welle ist unter anderem auf auf die Truppenbewegungen, vorallem dem Überqueren des Atlantiks durch Streitkräfte der Vereinigten Staaten, der Soldaten im ersten Weltkrieg, welcher 1918 sieben Monate nach der Entdeckung der spanischen Grippe endete, zurückzuführen. 
Die Ende August beginnende zweite Welle stellte nicht nur Krankenhäuser und Ärzte, sondern auch die Wirtschaft vor eine riesige Herausforderung. Während die erste Welle nur circa drei bis fünf Millionen Menschen tötete sorgte die zweite Welle mit bis zu 50 Millionen Todesopfern für globales Chaos. In Europa lag der Fokus der Presse jedoch nach wie vor auf dem ersten Weltkrieg lag und die Versorgungslage befand in vielen Staaten in einem sehr schlechten Zustand. In Indien, stark beeinflusst durch eine Hungersnot,  wird die Letalitätsrate während der zweiten Welle auf bis zu fünf Tote je einhundert Einwohnern und die Anzahl der Toten auf bis zu 20 Millionen geschätzt.\footnote{Vgl. \cite{Bax20}} Gründe für die hohe Verbreitungsgeschwindigkeit der zweiten Welle waren weiterhin reisende Soldaten, die Vermeidung von Quarantänisierung infizierter Personen während des Krieges und aufgrund überfüllter Krankenhäuser und dem Fehlen eines wirksamen Gegenmittels.
Ab Februar 1919 gab es in vielen Staaten erneut eine, weitaus harmlosere, Grippewelle, welche aufgrund der sehr jungen Todesopfer ebenfalls der Spanischen Grippe zugerechnet wird. Da der Virus ab dem Jahre 1920 deutlich weniger tödlich und Grippewellen der folgenden Jahre von weitaus geringerem Ausmaß und anderer Altersverteilung geprägt waren gelten diese Jahre als das Ende der Pandemie.
\subsubsection{Reaktionen und Gegenmaßnahmen}
Bereits mit Ende der ersten Welle leiteten erste Länder Quarantänemaßnahmen ein, Zeitungen warnten vor Symptomen und Doktoren rieten zu Abstand und Hygiene. Jedoch waren beispielsweise Quarantäneanordnungen von Schiffen aufgrund des Krieges unumsetzbar, Verschwörungstheorien und der Fakt, dass keine wirkende Impfung gegen die spanische Grippe existierte erschwerten den Prozess des Aufhaltens der Pandemie ebenfalls.\footnote{Vgl. \cite{Hist10}} Global verbreiteten sich Empfehlungen um eine Infektion zu vermeiden. So empfahl die Zeitung "Douglas Island News" im November 1918 beispielsweise Menschenmengen zu meiden, häufig zu lüften, eine Maske zu tragen, sich warm zu halten und den Mund nicht zu berühren.\footnote{Vgl. \cite{Mai2021}} Obwohl spätere Studien zeigten, dass die Mund-Nasen-Schutzpflicht und das Verbot von Massenveranstaltungen die Todesrate in diversen amerikanischen Großstädten um bis zu 50 Prozent senkte steißen diese Einschränkungen nicht in allen Teilen der Bevölkerung auf Zustimmung.
\subsubsection{Folgen}

\subsection{Zur COVID-19-Pandemie}
Die erstmals im Dezember 2019 in Wuhan beschriebene Infektionskrankheit, welche durch den Virus SARS-CoV-2 hervorgerufen wird, infizierte innerhalb von zwei Jahren mehr als 300 Millionen Menschen mit Fällen in jedem derzeit existierenden Staat. Trotz einer deutlich geringeren Tödlichkeit als die Spanische Grippe erlagen bereits 5,5 Millionen Menschen der Krankheit weltweit. 
\subsubsection{Krankheitsverlauf und Symptome}
 Nach einer durchschnittlichen Inkubationszeit von circa fünf bis sechs Tagen treten meist Symptome wie Husten, Schnupfen, Fieber, Gliederschmerzen, Geruchs- und Geschmacksverlust und Abgeschlagenheit auf. Sowohl in der Art der Hauptsymptome als auch in der großen Anzahl der weniger häufigen Krankheitserscheinungen gleichen sich Covid 19 und die Spanische Grippe sehr.\footnote{Vgl. \cite{Gov21}} Die genannten Symptome klingen in circa 81 Prozent der Fälle innerhalb von zwei Wochen ohne größere Komplikationen wieder ab, bei schwereren Verläufen entwickelt sich jedoch in der zweiten Krankheitswoche aufgrund einer Infektion der unteren Atemwege bis zur Lungenentzündung eine schwere Atemnot. In diesem Fall kann die 
\subsubsection{Ausbreitung und Verlauf}
Auch der SARS-CoV-2 Virus wird über die in 2.2.1 beschriebene Tröpfcheninfektion übertragen, die Möglichkeit einer Schmierinfektion wurde zum Zeitpunkt des Verfassens dieser Arbeit vom Robert-Koch-Institut noch nicht ausgeschlossen. 
\subsubsection{Reaktionen und Gegenmaßnahmen}
\subsubsection{Folgen}
\subsection{Direkter Vergleich }
erst bei Temperaturen oberhalb von 22 °C verringert sich diese deutlich
\section{Herausstellung der beeinflussenden Faktoren}
\section{Fazit und Zusammenfassung}
\section{TODO}
- Fußnoten -> Name, Vorname und Statista Internetnutzer ersteller einsetzen
- fehler fixen
- fehlende Fußnoten ergänzen

\newpage
\printbibliography
\end{document}